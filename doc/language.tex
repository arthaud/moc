\documentclass[a4paper]{article}

\usepackage[french]{babel}
\usepackage{fontspec}

% {{{ minted
\usepackage{minted}
\newminted[moc]{cpp}{%
  tabsize=4, fontsize=\footnotesize,
  frame=lines, framesep=\fboxsep,
  rulecolor=\color{gray!40}
}
\newmintinline[imoc]{cpp}{}
% }}}

\author{
       Maxime Arthaud
  \and Korantin Auguste
  \and Martin Carton
  \and Étienne Lebrun
}
\title{Documentation du langage moc}

\begin{document}
  \maketitle
  \tableofcontents
  \newpage

  Ce document présente la syntaxe du langage moc. Ce langage est proche du C89
  avec quelques éléments en moins et quelques extensions.

  \section{Variables et types}
    Les variables se déclarent comme en C avec la limitation d'une variable par
    instruction. Les types supportés sont:
    \begin{itemize}
      \item \imoc+int+;
      \item \imoc+bool+;
      \item \imoc+char+;
      \item les pointeurs;
      \item les structures;
      \item les tableaux.
    \end{itemize}

    \imoc+void+ est aussi supporté comme type de retour de fonction ainsi que
    \imoc+void*+ comme pointeur non-typé.

    Le typage est plus fort qu'en C, des casts supplémentaires sont nécessaires
    (entre entiers et pointeurs, ou pointeurs de différents types par exemple).

    \subsection{Structures}
      Les structures sont supportées mais pas \imoc+typedef+\footnote{Qui
      rendrait la syntaxe contextuelle}. Leur définition est la même qu'en C:

      \begin{moc}
struct point {
    int x;
    int y;
};

int manhattan(struct point* p) { // struct obligatoire
    return p->x+p->y;
}
      \end{moc}

      Il n'est pas possible de déclarer une structure sans la définir (pas de
      \emph{forward declaration}).

      Les structures ne peuvent pas être copiées (donc elles peuvent pas être
      assignées, passées en paramètre ou retournées):

      \begin{moc}
int manhattan(struct point p) { // pas possible
    return p.x+p.y;
}

struct point origin() { // pas possible
    struct point o;
    o.x = o.y = 0;
    return o; // pas possible
}

void test() {
    struct point A;
    struct point B = A; // pas possible
    B = A; // pas possible
}
      \end{moc}

    \subsection{Tableaux}
      La syntaxe de déclaration des tableaux est différente de celle du
      C\footnote{Afin de simplifier la gestion des types dans la grammaire}:
      \begin{moc}
int[42] mon_tableau;

// au lieu de:
int mon_tableau[42]; // erreur syntaxique
      \end{moc}

  \section{Opérateurs}
    Le langage possède les mêmes opérateurs que le C avec les mêmes priorités à
    l'exception de:
    \begin{itemize}
      \item l'opérateur ternaire \imoc+?:+;
      \item les opérateurs \mintinline{tex}+@=+ (\imoc.+=., \imoc.-=., etc.);
      \item les opérateurs \imoc.++. et \imoc.--.;
    \end{itemize}

    Les opérateurs \imoc+/+ et \imoc+%+ ne sont pas supportés car CRAPS ne
    possède pas ces opérations.

    L'opérateur \imoc+sizeof+ ne fonctionne que sur un type (\imoc+sizeof a+ ne
    marche pas, par contre \imoc+sizeof(int)+ marche).

    Les opérateurs \imoc+||+ et \imoc+&&+ ne sont pas \emph{short-circuit}.

  \section{Conditions}
    Le \imoc+if+ du C est supporté et devrait fonctionner à peu près de la même
    façon.

    \imoc+switch+ et l'opérateur ternaire \imoc+?:+ ne sont pas supportés.

    Il n'est pas possible de déclarer une variable dans la condition du
    \imoc+if+.

  \section{Boucles}
    Le langage possède les boucles \imoc+while+ et \imoc+for+ du C. La boucle
    \imoc+do {} while+ et \imoc+goto+ ne sont pas supportés.

    Pour la boucle \imoc+for+, les trois parties sont obligatoires et la
    première partie de permet pas de déclarer une variable (seulement la
    définir). Exemple:

    \begin{moc}
int i;
for (i = 0; i < 10; i = i+1) {
    // ...
}
    \end{moc}

    Dans une boucle, \imoc+break+ et \imoc+continue+ peuvent être utilisés.
    Cependant si \imoc+break+ et \imoc+continue+ sont utilisés, toutes les
    variables doivent être déclarées en début de fonction. Le compilateur ne
    fourni ni warning, ni erreur en cas d'utilisation de variables déclarées au
    milieu de fonction et de saut dans une même fonction.

  \section{Fonctions}
    La syntaxe de définition d'une fonction est la même qu'en C.

    Il n'est pas possible de déclarer une fonction sans la définir (pas de
    \emph{forward declaration}). Les fonctions mutuellement récursives sont donc
    impossibles. Une fonction peut par contre être récursive.

    Une fonction peut être marquée \imoc{export}. Ce mot clé indique au
    compilateur qu'il ne doit pas supprimer la fonction si elle ne lui semble
    pas utile. Il est utile si une fonction n'est appelée que dans de
    l'assembleur \emph{inline}. La fonction \imoc+main+ est implicitement
    marquée \imoc+export+.

  \section{Asm}
    Il est possible d'inclure de l'assembleur \emph{inline} dans une fonction ou
    dans le contexte global. Ce code n'est pas vérifié.

    \begin{moc}
asm("set 0x4000, %sp");
    \end{moc}

    Cet instruction est utile pour le code d'initialisation.

    Il est possible de faire référence à une variable déclarée en C depuis
    l'assembleur en préfixant son nom par \verb+$+:
    \begin{moc}
int a;
asm("st %r20, $a");
    \end{moc}

    Il est possible d'appeler une fonction en préfixant son nom par \verb+f_+:
    \begin{moc}
void dummy1() {
}

int dummy2() {
    asm("
        push %r28
        call f_dummy1 // appelle dummy1
        pop %r28
    ");
}
    \end{moc}

  \section{Préprocesseur}
    Le compilateur utilise \verb+cpp+, le préprocesseur fourni avec \verb+gcc+.
    Tout ce qu'il est possible de faire en C avec le préprocesseur est donc
    possible dans ce langage (notamment \imoc+#include+ et \imoc+#define+).

\end{document}
