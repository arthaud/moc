\documentclass{scrartcl}
\usepackage[utf8x]{inputenc}
\usepackage[frenchb]{babel}
\usepackage{amssymb}
\usepackage{lmodern}
\usepackage[T1]{fontenc}
\usepackage{hyperref}
\usepackage{minted}

\usepackage{fancyhdr}
\pagestyle{fancy}
\renewcommand{\headrulewidth}{1pt}
\renewcommand{\footrulewidth}{\headrulewidth}

\title{Rapport projet de TDL}
\author{Korantin Auguste \and Etienne Lebrun \and Maxime Arthaud}
\date{mai 2014}

\newcommand{\mocc}{\texttt{mocc}}
\newcommand{\tam} {\textit{tam}}
\newminted[moccode]{cpp}{%
  tabsize=4, fontsize=\footnotesize,
  frame=lines, framesep=\fboxsep,
  rulecolor=\color{gray!40}
}

\begin{document}
  \maketitle
  \tableofcontents
  \newpage

\section{Introduction}
    Nous avons décidé de s'éloigner un peu du sujet, puisque nous
    n'avons pas utilisé EGG, mais une modification de ce dernier faite
    par Maxime (avec de nombreux ajouts intéressants, mais une syntaxe
    légèrement différente, et donc incompatible).
    De plus, nous avons souhaité non pas compiler le code seulement en TAM,
    mais aussi en x86.

    Notre compilateur est modulaire et contient donc des classes « machines » capables de générer du code
    assembleur. Nous avons donc une machine TAM, et une machine x86.

    Ainsi, nous pouvons compiler les programmes en assembleur x86, et les faire tourner sur nos
    propres machines !

    Nous disposons aussi d'un petit système de préprocesseur permettant d'inclure d'autres fichiers.
    
    Pour pouvoir s'amuser un peu, nous avons aussi développé une librairie TAM qui propose quelques
    fonctions utiles, pour afficher du texte... en appelant une simple fonction (au lieu de mettre
    du code assembleur).
    De même, nous disposons d'une petite librairie x86, développée entièrement par nos soins et qui
    effectue des appels systèmes (via une interruption processeur), afin de lire, d'écrire, de faire
    de l'allocation dynamique... Nous avons même codé une petite implémentation de la commande « cat »
    en MOC.
 
\section{EGG}

erreur quand init n'est pas déclaré

tds 

bla bla bla

\section{La machine TAM}

\subsection{Partie 1}

\subsubsection{gestion des chaînes de caractères}

Un des premiers problèmes que nous avons eu à résoudre est la gestion des chaînes de caractères.
Nous voulions en effet qu'elles soient compatibles avec les char*, ce qui n'est pas possible 
avec les fonctions proposées par TAM, qui ne gèrent que des chaînes statiques.
La règle  F -> string appelle donc la  fonction genString() de la classe MTAM.
Celle ci génère un code qui sera executé au début du programme, et qui positionne la chaine en mémoire 
en début de pile avec des LOADL . Le code renvoyé par genString() pose sur la pile un pointeur vers le 
premier caractère de la chaîne.

\subsubsection{sous-classes TAM Parameters et Variable Locator}

Ces classes sont nécessaires car le traitement des variables et des paramètres diffère en fonction du langage cible.

Le parameters locator permet de positionner les paramètres lors de l'appel d'une fonction. Le premier paramètre se situe en -1[LB], 
les suivants sont en dessous.

Le variable locator est initialisé à partir d'un offset initial, et alloue les positions en décalant l'offset en fonction de la taille des variables.
À chaque nouveau bloc, un nouveau locator est généré à partir du précédent en lui communiquant son offset courant. De cette manière , les variables locales
au sous bloc pouront être écrasées par le premier bloc.
 

\subsubsection{lib TAM}

nous avons décidé d'écrire une petite librairie standard, principalement pour gérer les méthodes sur les 
chaînes de caractères et malloc. C'est elle aussi qui contient l'assembleur inline "CALL (LB) f\_main 
    HALT " qui permet de lancer le programme principal. Elle est donc inclue dans tous les tests de TAM à l'aide du #include qui est remplacé par le préprocesseur.
L'inclusion de la boucle while à la définition du langage a permis de la rédiger en grande partie en Moc.

\subsection{Partie 2}

La principale difficulté de cette partie 2 a été la gestion de la liaison tardive.
Nous avons donc implémenté un système de table de virtuels.
Le code qui génère la vtable est ajouté au code d'initialisation lors de la création de la classe correspondante.
Il est constitué d'autant de LOADA que de méthodes , suivis des labels correspondants, ceux de la classe mère pour 
les méthodes qui n'ont pas été redéfinies. Le premier champ d'une instance de classe est un lien vers la vtable de la classe réelle.
Lors d'un appel de méthode, on ajoute à celui-ci la position de la méthode voulue dans la vtable.
On charge la valeur avec un loadi puis on appelle la méthode avec un CALLI.
CALLI n'empilant que deux valeurs, on doit rajouter une valeur sur la pile avec un LOADL 0 .



\section{La machine x86}

La machine x86 est chargée de générer du code assembleur x86 (qui devra être compilé par nasm).
Au final, elle est très proche de la machine TAM et je ne m'étendrai pas sur le sujet.
Toutefois, un point différe complètement : là ou TAM est une machine à pile pure, en x86 nous
devons gérer les registres.

\subsection{Gestion des registres}

Plusieurs choix se sont offerts à nous pour gérer les registres en x86 :

Tout d'abord, nous aurions pu tout mettre sur la pile, à la façon de TAM. Lors d'un calcul,
il suffisait de faire comme TAM en effectuant des pop, de faire le calcul, et de tout remettre
sur la pile. Ce système aurait été très simple, mais nous aurions toujours utilisé un ou deux
registres, et notre but était d'aller un peu plus loin que la simple machine à pile.

A l'autre extrème, il aurait été possible dans l'idéal de stocker certaines variables
uniquement dans des registres, sans jamais les mettre en mémoire, et d'utiliser tous les
registres de manière intensive. C'est ce que font les vrais compilateurs, mais il s'agit
de quelque chose d'extrêmement complexe.

Au final, nous avons retenu une solution qui est relativement simple, tout en tirant déjà plus
partie des registres : les variables sont toutes stockées sur la pile, mais lors d'un calcul
tous les résultats intermédiaires pourront être stockés dans divers registres, et n'auront jamais
besoin d'être mis sur la pile.

Pour cela, nous utilisons une pile qui retient les registres contenant les résultats
des derniers calculs. Ainsi, un calcul verra ses résultats intermédiaires stockés dans différents
registres. Cette solution marche plutôt bien, tire vraiment partie des registres et n'a pas été
trop dure à mettre en oeuvre.

\subsection{lib x86}

Pour faciliter le travail, nous avons développé une petite bibliothèque qu'il suffit d'inclure,
et qui se charge de générer un squelette de programme en assembleur, avec une dizaine de 
fonctions utile pour permettre à nos programme de faire des choses : lire, écrire, quitter,
afficher du texte...

Son développement fût très intéressant, puisque nous avons tout fait nous même, sans faire
appel à une bibliothèque tierce comme la libc.
Pour communiquer avec le système d'exploitation, il suffit d'utiliser l'instruction « int »
permettant de faire une interruption, et de mettre les bonnes données dans les registres,
pour dire au système d'exploitation ce que l'on attend de lui.

Ainsi, pour afficher du texte nous avons une fonction print, qui fait elle-même appel à une fonction
write, qui effectue un appel système :
\begin{moccode}
int write(int fd, char* buf, int nbytes) {
    asm("
        ; write syscall
        mov eax, 4
        mov ebx, %fd
        mov ecx, %buf
        mov edx, %nbytes
        int 0x80 ; the result is returned into eax : perfect
    ");
}

void print(char *s) {
    write(1, s, strlen(s));
}
\end{moccode}


\section{Tests}

\section{Conclusion}

Ce projet aura été l'occasion d'effectuer un petit compilateur jouet,
et nous aura permis de ce rendre compte que c'était une tâche assez longue,
mais réalisable.

Toutefois, nous avons aussi mis le doigt sur le fait que s'il est simple
de réaliser un petit compilateur qui fonctionne, arriver à lui faire
générer du code rapide et optimisé doit être une tâche extrêmement difficile.


Le fait d'arriver à générer du x86 a donné un côté très amusant au projet,
puisque nous pouvions lancer nos programmes sur nos machines, et même
utiliser un debugger comme gdb pour comprendre certains bugs.

Arriver à effectuer des appels système pour avoir un programme qui fait
de vraies actions sur nos ordinateurs était aussi particulièrement sympathique !

\end{document}

